\documentclass[a4paper,norsk, 10pt]{article}
\usepackage[utf8]{inputenc}
\usepackage{verbatim}
\usepackage{listings}
\usepackage{graphicx}
\usepackage[norsk]{babel}
\usepackage{a4wide}
\usepackage{color}
\usepackage{amsmath}
\usepackage{float}
\usepackage{amssymb}
\usepackage[dvips]{epsfig}
\usepackage[toc,page]{appendix}
\usepackage[T1]{fontenc}
\usepackage{cite} % [2,3,4] --> [2--4]
\usepackage{shadow}
\usepackage{hyperref}
\usepackage{titling}
\usepackage{marvosym }
\usepackage{subcaption}
\usepackage[noabbrev]{cleveref}
\usepackage{cite}
\usepackage{todonotes}


\setlength{\droptitle}{-10em}   % This is your set screw

\setcounter{tocdepth}{2}

\lstset{language=c++}
\lstset{alsolanguage=[90]Fortran}
\lstset{alsolanguage=Python}
\lstset{basicstyle=\small}
\lstset{backgroundcolor=\color{white}}
\lstset{frame=single}
\lstset{stringstyle=\ttfamily}
\lstset{keywordstyle=\color{red}\bfseries}
\lstset{commentstyle=\itshape\color{blue}}
\lstset{showspaces=false}
\lstset{showstringspaces=false}
\lstset{showtabs=false}
\lstset{breaklines}
\title{STK4900 Oblig 2}
\author{Daniel Heinesen, daniehei}
\begin{document}
\maketitle

\section{Problem 1}
\subsection{a)}
We have a data set where the outcome is whether the female crab has one or more satellites $y = 1$, or none $y=0$. We are looking for a regression that can, given the covariants, give us a probability that the female crab have satellites. This means that we are looking for a regression model that gives us a probability for a binary outcome. The best choice for such a model is a logistic regression model
\begin{equation}
p(x_1,...,x_p) = \frac{\exp(\beta_0 + \beta_1 x_1 + ... + \beta_n x_n)}{1+\exp(\beta_0 + \beta_1 x_1 + ... + \beta_n x_n)},
\end{equation}
where $p$ is the desired probability, $x_i$ the covariates and $\beta_i$ their fitted coefficients\todo{Is that the best way to describe  $\beta_i$?}.

\subsection{b)}

We want to find the odds ratio between crabs that differ with one centimetre in with. We know that with width as the only covariant, we define the odds as 

\begin{equation}
\frac{p(x)}{1-p(x)} = \exp(\beta_0 + \beta_1 x).
\end{equation}
From this we get the odds ratio for a difference in one centimetre
\begin{equation}
OR = \frac{p(x+1)/[1-p(x+1)]}{p(x)/[1-p(x)]} = \frac{\exp(\beta_0 + \beta_1\cdot (x+1))}{\exp(\beta_0 + \beta_1 x)} = \exp(\beta_1\cdot 1) = \exp(\beta_1).
\end{equation}


\begin{table}[!ht]
\centering
\begin{tabular}{rrrrr}
  \hline
 & Estimate & Std. Error & z value & Pr($>$|$z$|) \\ 
  \hline
(Intercept) & -12.3508 & 2.6287 & -4.70 & 0.0000 \\ 
  width & 0.4972 & 0.1017 & 4.89 & 0.0000 \\ 
   \hline
\end{tabular}
\caption{Summary of the logical regression done the satellite crabs, with width as the only covariant.}\label{tab:crab_width}
\end{table}

We get the $\beta$'s from the logical regression found in table \ref{tab:crab_width}. From this we see that 
\begin{equation}
OR = \exp(0.4972) = 1.64.
\end{equation}
This means that odds of a female crab having satellite males increase by $64\%$ if the width of female increases by $1$ cm.

If $p(1) = p(x+1)$ and $p(0)=p(x)$ are small, we can approximate the odds ratio with the relative risk $RR = p(1)/p(0)$. In this case we have that $p(1) = 6.8\cdot 10^{-6}$ and $p(0) = 4.1 \cdot 10^{-6}$, which means that both are very small. This means that we can assume that $RR \approx OR$. And comparing them $OR = 1.6367$ and $RR = 1.6487$ we see that this is correct.

Since we know that 
\begin{equation}
t = \frac{\beta}{se(\beta)}
\end{equation}
is close to normally distributed, we can use this to find a confidence interval for the odds ratio. We simply do this by calculating the confidence interval for $\beta_1$ and then taking $\exp$ of this interval. 

\begin{table}[!ht]
\centering
\begin{tabular}{rrrr}
  \hline
 & expcoef & lower & upper \\ 
  \hline
(Intercept) & 4.33e-06 & 2.50-08 & 0.00075 \\ 
  width & 1.64 & 1.35 & 2.01 \\ 
   \hline
\end{tabular}
\caption{Confidence interval for the odds ratio for crabs differing by one cm in width.}\label{tab:crabs_OR_CI}
\end{table}

From tab. \ref{tab:crabs_OR_CI} we see that we get a confidence interval $CI = (1.35,2.01)$. Since $1$ is outside of this interval, we can say that width gives an significant increase in the probability of satellites. 

\subsection{c)}

\subsection{d)}
\subsection{e)}


\section{Problem 2}
\subsection{a)}
We have an outcome, medals, which is a count outcome, and we therefore assume that the count is distributed with a Poisson distribution $Y_i \sim Po(\lambda_i)$. Such a distribution is parametrized with a parameter $\lambda$, the rate. In our data we will have that this rate is dependent on several covariants, so we need a way to determine this dependence on the covariants. It is here we use Poisson regression. Given $n$ independent subjects, we have that 
\begin{itemize}
\item $y_i$ is the count of the $i^{th}$ subject
\item $x_{ij}$ is the $j^{th}$ covariant for the $i^{th}$ subject
\end{itemize}
We then define our model as 
\begin{equation}
\lambda_i = \lambda(x_{1,i},...,x_{p,i}) = \exp(\beta_0 + \beta_1 x_{1,i} + ... + \beta x_{p,i}).
\end{equation}

It is this model we want to fit to the medal counts. But there is something we have to be careful with: A country with a higher number of athletes will most likely have more medals than a country with fewer athletes. So instead of the medal count following the distribution $Y_i ~ Po(\lambda_i)$ we instead say that they follow the distribution $Y_i ~ Po(w_i\lambda_i)$, where $w_i$ is the number of athletes representing the country. But how do use this in our model? We can see this from taking the expected value

\begin{equation}
E[Y_i] = w_i \lambda_i = w_i\exp(\beta_0 + \beta_1 x_{1,i} + ... + \beta x_{p,i}) = \exp(\log(w_i) + \beta_0 + \beta_1 x_{1,i} + ... + \beta x_{p,i}).
\end{equation}

This means that to compensate for this imbalance of athletes, we can fit out model with $\log(w_i)$ as a covariant. This is what we call an offset. This we already $Log.athletes$ in our data, we can just use this as the offset in our regression.

\subsection{b)}
To find a fit for our model we are going to use two methods. The first is to fit a model with all the covariant and see which of them are significant. The other method is to add one covariant after an other and use a two-way ANOVA to see if the addition is significant.

\begin{table}[ht]
\centering
\begin{tabular}{rrrrr}
  \hline
 & Estimate & Std. Error & z value & Pr($>$|$z$|) \\ 
  \hline
(Intercept) & -2.8623 & 0.3191 & -8.97 & 0.0000 \\ 
  Log.population & 0.0275 & 0.0315 & 0.87 & 0.3831 \\ 
  GDP.per.cap & -0.0149 & 0.0032 & -4.65 & 0.0000 \\ 
  Total1996 & 0.0118 & 0.0016 & 7.36 & 0.0000 \\ 
   \hline
\end{tabular}
\caption{Summary of a Poisson regression with all the covariants. $Log.popilation$ is the logarithm of the nation's population size per 1000, $GDP.per.cap$ is the GDP per capita and $Total1996$ is the medal count for the previous Olympic Games.}\label{tab:ol_full_it}
\end{table}


From table\ref{tab:ol_full_it} we see the summary of the fit with all covariant. We see that with p-values close to $0$




\section{Problem 3}







\end{document}

